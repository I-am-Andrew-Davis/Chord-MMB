% Allows compilation of just this section in this directory
\ifx\havefulldoc\undefined
  \newif\iffulldoc
\ifx\havefulldoc\undefined
  \fulldocfalse
\else
  \fulldoctrue
\fi
\documentclass[11pt]{article}

%\usepackage{subfigmat}  % matrices of similar subfigures, aka small mulitples
%                        % use this instead of subfig which can cause conflicts  
\usepackage{graphicx}
\usepackage[rgb]{xcolor}
\usepackage{calc}        % Can add lengths
\usepackage{ifthen}
\usepackage{tikz}
\usepackage{pgfplots}
\usepackage{xspace}      % Spacing after some new commands
\usepackage{amsmath}
\usepackage{amssymb}
\usepackage{bm}
\usepackage{cancel}      % show math cancellation
\usepackage{subfig}
\usepackage{wrapfig}     % Wrap text around figures
%%\usepackage{sidecap}   % Caption on side
\usepackage{caption}     % Caption format
\usepackage[numbers,sort&compress]{natbib}
\usepackage{url}         % Can use \path for textt
%%\usepackage{ctable}    % Tables with footnotes
\usepackage{array}       % Table columns
\usepackage{longtable}   % Table with page breaks
\usepackage{newtxtext,newtxmath}
%\usepackage{mathptmx}    % Times new roman font
%\usepackage{mathpazo}  % Palatino font
\usepackage[hypertexnames=false, pdfborder={0 0 0}]{hyperref}  % Links in the document
\usepackage[pass,paperwidth=8.5in,paperheight=11in]{geometry}  % Papersize
\usepackage{enumitem}
\usepackage{pdfpages}
\usepackage{listings}    % For code listing
\usepackage{verbatim}    % For verbatim input

\pgfplotsset{compat=1.5}

%--Page formatting

\setlength{\topmargin}{0.in}
\setlength{\headheight}{0.in}
\setlength{\headsep}{0.in}
\setlength{\oddsidemargin}{0.in}
\setlength{\evensidemargin}{0.in}
\setlength{\textwidth}{6.5in}
\setlength{\textheight}{9in}
\setlength{\footskip}{10mm}

%--Caption formatting

\captionsetup{font=small,margin=10pt,maxmargin=0.1\linewidth,labelfont=bf,labelsep=quad}

%--Hyperref setup

% All black
\hypersetup {
colorlinks= true,
urlcolor=black,
citecolor=black,
linkcolor=black,
}

%--Source listings formatting

\lstset{language=C++,
  basicstyle=\ttfamily,
%  keywordstyle=\bfseries  % Not necessary for only bold
  columns=fixed,
  basewidth=0.5em,         % ~Same column width as verbatim
%  numbers=left,
  numberstyle=\scriptsize,
%  showstringspaces=false,
  morekeywords={shared,private}
}
%\lstset{
%        language=C++,
%        basicstyle=\scriptsize,
%%       numbers=left,
%%       numberstyle=,
%        stepnumber=1,
%        numbersep=5pt,
%        backgroundcolor=\color{white},
%        showspaces=false,
%        showstringspaces=false,
%        showtabs=false,
%        frame=single,
%        tabsize=2,
%        captionpos=b,
%        breaklines=true,
%        breakatwhitespace=false,
%        escapeinside={\%*}{*)}
%}

%%%%%%%%%%%%%%%%%%%%%%%%%%%%%%%%%%
%%                              %%
%% New commands                 %%
%%                              %%
%%%%%%%%%%%%%%%%%%%%%%%%%%%%%%%%%%

\makeatletter

%--Preliminary environment

\newenvironment{preliminary}%
  {\pagestyle{plain}\pagenumbering{roman}}%
  {\cleardoublepage\pagenumbering{arabic}}

%--Roman numerals

\newcommand{\rmnum}[1]{\romannumeral #1}
\newcommand{\Rmnum}[1]{\expandafter\@slowromancap\romannumeral #1@}

%--Create a \deg command

\DeclareMathSymbol{\circx}{\mathord}{symbols}{"0E}
\let\olddeg=\deg
\renewcommand\deg{\ensuremath{\,^\circx\,}}

%--A nice C++ symbol

\newcommand*{\Cpp}{C\parbox{1em}{\vspace*{-0.6ex}$\scriptstyle\mspace{-1mu}+\mspace{-1mu}+$}\xspace}%

%--This section creates the \un command for units and \E for exponents
%--E.g., 2.9\un{m^{2}} or 2.9\E{-3}\un{m^{2}}
%--Can be used in or out of math environment (math ensured in braces)
%--also a \un* for no spacing

\DeclareMathSymbol{\timesx}{\mathord}{symbols}{"02}
\newcommand{\etimes}{\mskip1.1mu\timesx\mskip0.9mu\relax}
\newcommand{\E}[1]{\ensuremath{\etimes10^{#1}}}

\newmuskip\unitmuskip
\newmuskip\unitpreskip
\unitmuskip=1mu plus 0.3mu              % Change this to suit your preference
\unitpreskip=4mu plus 1mu minus 1mu     % Change this to suit your preference
\def\unitmskip{\penalty10000 \mskip\unitmuskip}%
\DeclareRobustCommand\un{\@ifstar{\@unstar}{\@un}}
\newcommand\@unstar[1]{\leavevmode
  \ensuremath{%
    \begingroup \fam\z@%
    \let\,\unitmskip \unit@PreserveSpaces\@empty #1 \unit@PreserveSpaces%
    \endgroup}}%
\newcommand\@un[1]{\leavevmode%
  \unskip\penalty1000%
  \ensuremath{%
    \@checkfordeg #1\@empty%
    \begingroup \fam\z@%
    \let\,\unitmskip \unit@PreserveSpaces\@empty #1 \unit@PreserveSpaces%
    \endgroup}}
\def\unit@PreserveSpaces#1 {#1\@ifnextchar\unit@PreserveSpaces{\@gobble}%
  {\unitmskip \unit@PreserveSpaces\@empty}}%
\def\@checkfordeg{\@ifnextchar\deg{\@junkrest}{\mskip\unitpreskip\@junkrest}}%
\def\@junkrest#1\@empty{}%

%--Corrections and notes

\newcommand{\mnote}[1]{\marginpar{\textcolor{red}{#1}}}%
\newcommand{\inote}[1]{\textcolor{red}{[#1]}}%
\newcommand{\fnote}[1]{\footnote{\textcolor{red}{#1}}}%

%--Compact enumeration

\newcounter{cmpctenum}
\newenvironment{compactenum}{\begin{list}{\arabic{cmpctenum}.}
  {\usecounter{cmpctenum}
   \setlength{\parsep}{0pt}
   \setlength{\itemsep}{0.5ex}}
  }
{\end{list}}

%--Compact itemize

\newenvironment{compactitem}{\begin{list}{{\footnotesize$\bullet$}}
  {%\setlength{\topsep}{0pt}
   \setlength{\parsep}{0pt}
   \setlength{\itemsep}{0.5ex}
   \setlength{\leftmargin}{1.5em}
   \setlength{\itemindent}{1.0em}}
  }
{\end{list}}

%--Compact itemize for function arguments

\newenvironment{argitem}{\begin{list}{{\footnotesize$\circ$}}
  {\setlength{\topsep}{0pt}
    \setlength{\parsep}{0pt}
   \setlength{\itemsep}{0.5ex}
   \setlength{\leftmargin}{2em}
   \setlength{\itemindent}{-1em}}
  }
{\end{list}}

%--Figure location which changes if full or chapter compilation

\iffulldoc
  \newcommand*{\figpre}[1]{figures/#1}
  \newcommand*{\drawpre}[1]{drawings/#1}
\else
  \newcommand*{\figpre}[1]{../figures/#1}
  \newcommand*{\drawpre}[1]{../drawings/#1}
\fi

%--Consistent references

\newcommand{\figref}[1]{Figure~\ref{#1}} 
\newcommand{\tabref}[1]{Table~\ref{#1}} 
\newcommand{\secref}[1]{Section~\ref{#1}} 
\newcommand{\chapref}[1]{Chapter~\ref{#1}} 
\newcommand{\appref}[1]{Appendix~\ref{#1}} 

%--Commonly used math/Chombo notation

\newcommand{\ibold}{{\boldsymbol{i}}}
\newcommand{\jbold}{{\boldsymbol{j}}}
\newcommand{\kbold}{{\boldsymbol{k}}}
\newcommand{\ebold}{{\boldsymbol{e}}}
\newcommand{\pbold}{{\boldsymbol{p}}}
\newcommand{\ubold}{{\boldsymbol{u}}}
\newcommand{\vbold}{{\boldsymbol{v}}}
\newcommand{\xbold}{{\boldsymbol{x}}}
\newcommand{\bbold}{{\boldsymbol{b}}}
\newcommand{\hbold}{{\boldsymbol{h}}}
\newcommand{\Ubold}{{\boldsymbol{U}}}
\newcommand{\Wbold}{{\boldsymbol{W}}}
\newcommand{\Nbold}{{\boldsymbol{N}}}
\newcommand{\Xbold}{{\boldsymbol{X}}}
\newcommand{\xibold}{{\boldsymbol{\xi}}}
\newcommand{\nablabold}{{\boldsymbol{\nabla}}}
\newcommand{\Deltabold}{{\boldsymbol{\Delta}}}
\newcommand{\phibold}{{\boldsymbol{\phi}}}
\newcommand{\eboldd}{{\ebold^d}}
\newcommand{\zerobold}{{\boldsymbol{0}}}
\newcommand{\Dim}{{\mathbf{D}}}
\newcommand{\NboldT}{{\Nbold^T}}
\newcommand{\NrmT}{{\text{N}}^T}
\newcommand{\Nrm}{{\text{N}}}
\newcommand{\Frm}{{\text{F}}}
\newcommand{\Wrm}{{\text{W}}}
\newcommand{\Urm}{{\text{U}}}
\newcommand{\Fbrm}{{\text{\bfseries F}}}
\newcommand{\Gbrm}{{\text{\bfseries G}}}
\newcommand{\Sbrm}{{\text{\bfseries S}}}
\newcommand{\Ubrm}{{\text{\bfseries U}}}
\newcommand{\Wbrm}{{\text{\bfseries W}}}
\newcommand{\Nc}{{\mathcal{N}}}
\newcommand{\vecnabla}{{\vec{\nabla}}}
\newcommand{\vecx}{{\vec{x}}}
\newcommand{\vecxi}{{\vec{\xi}}}

% Full partial {numerator}{denominator}{degree}
\newcommand*{\fpartial}[3]{\ifnum#3=1\ensuremath{\frac{\partial #1}{\partial #2}}\else\ensuremath{\frac{\partial^{#3} #1}{\partial #2^{#3}}}\fi}
% Full partial {numerator}{denominator}{degree} with extra spacing for denom
\newcommand*{\fpartialspc}[3]{\ifnum#3=1\ensuremath{\frac{\partial #1}{\partial #2}}\else\ensuremath{\frac{\partial^{#3} #1}{\partial #2^{\,#3}}}\fi}
% Full mixed partial {numerator}{denom1}{degree1}{denom2}{degree2}
\newcounter{npdeg}
\newcommand*{\fmpartial}[5]{\setcounter{npdeg}{#3}\addtocounter{npdeg}{#5}\ensuremath{\frac{\partial^{\thenpdeg} #1}{\partial{#2}\ifnum#3>1^{#3}\fi\partial{#4}\ifnum#5>1^{#5}\fi}}}
% Cell step subscript (argument is '+n'+ or '-n')
\newcommand*{\hstep}[1]{\ensuremath{{\ibold#1\eboldd}}}
% Cell half step subscript (argument is '+' or '-')
\newcommand*{\hhstep}[1]{\ensuremath{{\ibold#1\frac{1}{2}\eboldd}}}
% Cell half step subscript (argument is '+' or '-'), second is number of halves
\newcommand*{\hhstepn}[2]{\ensuremath{{\ibold#1\frac{#2}{2}\eboldd}}}
% Average angle brackets
\newcommand*{\avg}[1]{\ensuremath{\langle #1\rangle}}

% \vt = v_\theta
\newcommand*{\vt}{\ensuremath{v_\theta}}

%--Spacing

\newcommand{\zapspace}{\topsep=1pt\partopsep=1pt\itemsep=1pt\parskip=2pt}

\makeatother

%--Float fractions

\renewcommand{\topfraction}{0.9}
\renewcommand{\bottomfraction}{0.9}
\renewcommand{\textfraction}{0.1}
\renewcommand{\floatpagefraction}{0.9}

  \begin{document}
  \fi

\begin{center}
  \large\bfseries Mapped operations using transformations from computational to
  physical space
\end{center}

%% Define commands for consistent treatment throughout document
\newcommand{\eqnref}[1]{\text{Eq.}~(\ref{eq:#1})}
% \newcommand{\eqnsref}[1]{\text{Eqs.}~(\ref{eq:#1})}
% \newcommand{\afigref}[1]{\text{Fig.}~\ref{fig:#1}}
% \newcommand{\figsref}[1]{\text{Figs.}~\ref{fig#1}}
% \newcommand{\tblref}[1]{\text{Table}~\ref{tbl:#1}}
% \newcommand{\class}[1]{\texttt{#1}}
% \newcommand{\package}[1]{\texttt{#1}}
% \newcommand{\file}[1]{\textt{#1}}
% \newcommand{\BibTex}{\textsc{Bib}\Tex}

% \newcommand{\NrmT}{{\rm{N}}^{T}}
\newcommand{\NtJ}{\frac{{\NrmT}}{J}}
\newcommand{\NtJs}[1]{\frac{{\NrmT}_{#1}}{J}}
\newcommand{\NJ}{\frac{{\Nrm}}{J}}
\newcommand{\rmT}{{\rm T}}

\usetikzlibrary{calc}
\usetikzlibrary{patterns}

% \maketitle

% \section*{Notation}
% \begin{multicols}{2}
%   \begin{tabbing}
%     XXXXX \= \kill %set tab stop
%     $\vec{\Fbrm}$ \> flux dyad, e.g., [${\Fbrm_1,\cdots, \Fbrm_D}$] \\
%     $\Fbrm$ \> flux vector, e.g., [$\Frm_1, \cdots, \Frm_N$] \\
%     $\Ubrm$ \> solution vector, e.g., [$\Urm_1, \cdots, \Urm_N$] \\
%     $\Frm_j$, ${\rm{G}}_j$ \> the $j^{th}$ component of vector ${\Fbrm}$ (or \Gbrm) \\
%     $\NrmT$ \> metric transformation matrix \\
%     $J$ \> grid metric Jacobian \\
%     $\ibold$ \> grid indices, e.g., ($i, j, k$) in 3D \\
%     $\eboldd$ \> unit vector in direction $d$ \\
%     $\avg{\cdot}$ \> the cell-averaged or face-averaged \\
%     $\vecx$ \> physical space, e.g., ($x, y, z)$ \\
%     $\vecxi$ \> computational space, e.g., ($\xi, \eta, \zeta$) \\
%     $\vecnabla_x$ \> $\vecnabla$ in physical space\\
%     $\vecnabla_\xi$ \> $\vecnabla$ in computational space
%   \end{tabbing}
% \end{multicols}

% \section{Mapping in the Fourth-Order Accurate Finite-Volume Scheme}
% To provide a brief background for how mapping is used in Chord, the major steps are reviewed for a fourth-order finite-volume method with generalized curvilinear transformation. More thorough details may be found in previous papers by Guzik et al.\cite{Guzik2015, Guzik2012}, with the main points summarized here.

% \subsection{The Conservation Laws}
% The finite-volume method for Cartesian grids is formulated using the integral form of the conservation laws with control volumes $V_i$ in physical space, $\vecx$,
% \begin{equation*}
%   \fpartial{}{t}{1}\int_{\vecx(V_{\ibold})} \Ubrm\,\mathrm{d}\vecx + \int_{\vecx(V_{\ibold})}\vecnabla_{x}\cdot\vec{\Fbrm}\,\mathrm{d}\vecx = 0\,.
% \end{equation*}
% %%
% Transforming this into computation space, $\vecxi$, is done by using grid metric terms such that
% %%
% \begin{align*}
%   &\fpartial{}{t}{1}\int_{V_{\ibold}} J\Ubrm\,\mathrm{d}\vecxi + \int_{V_{\ibold}} (J \vecnabla_x \vecxi \vecnabla_{\xi})\cdot \vec{\Fbrm}\,\mathrm{d}\vecxi = 0 \\
%   &\fpartial{}{t}{1}\int_{V_{\ibold}} J\Ubrm\,\mathrm{d}\vecxi + \int_{V_{\ibold}} \vecnabla_{\xi}\cdot(\NrmT\vec{\Fbrm})\,\mathrm{d}\vecxi = 0
% \end{align*}
% where the transformation matrix, $\NrmT$, describes the grid metrics, and the metric Jacobian is defined by $J \equiv \det ( \vecnabla_{\xi} \vecx)$. It is of note here that although the divergence operator is applied to $\NrmT$, it is indeed divergence free.
% %Given that $\NrmT = J  \vecnabla_{x} \vec\xi$, with $\sum_j \frac{\partial\Nrm_{i,j}}{\partial \xi_j} = 0$.
% After applying the divergence theorem of Gauss, the integrals can be represented as cell averaged values, and the hyperbolic fluxes yield
% %%
% \begin{gather}
%   \frac{\mathrm{d}}{\mathrm{d}t}\avg{J\Ubrm}_\ibold + \frac{1}{h}\sum_{d=1}^D\left(\avg{\NrmT_d\vec{\Fbrm}}_\hhstep{+} - \avg{\NrmT_d\vec{\Fbrm}}_\hhstep{-}\right) = 0\,,\label{eq:semidiscrDiv.map}
% \end{gather}
% %%
% where the subscript $d$ denotes the $d^{th}$ row of $\NrmT$. The algorithm for arriving at a freestream-preserving $\avg{\NrmT_d\vec{\Fbrm}}$ from $\NrmT_d\vec{\Fbrm}$ is a rather involved process, and so is excluded in this setting but can be found in work by Guzik et al.\cite{Guzik2015}.
% It suffices to say that the mapping $\vecx(\vecxi)$ and its inverse are required on codimension two elements of the grid (e.g., vertices in 2-D and edges in 3-D).

%   For a single solution variable, the averaged values in equation~\eqref{eq:semidiscrDiv.map} can be obtained at fourth-order accuracy from
% %%
% \begin{equation}
%   \avg{\NrmT_d\vec{\mathrm{F}}}_\hhstep{\pm} = \sum_{s=1}^D\avg{\NrmT_{d,s}}_{\hhstep{\pm}}\avg{\mathrm{F}_{s}}_{\hhstep{\pm}} + \frac{h^2}{12}\sum_{s=1}^D\sum_{d'\ne d}\fpartial{\NrmT_{d,s}}{\xi_{d'}}{1}\fpartial{\mathrm{F}_{s}}{\xi_{d'}}{1} + O(h^4)\,.
%   \label{eq:fspmappedflux}
% \end{equation}

% where
% %%
% \begin{equation}
%   \avg{\NrmT_{d,s}} = \frac{1}{h^{D-1}}\sum_{\pm=[+,-]}\sum_{d'\ne d}\pm\int_{l_{d,d',\pm}}\Nc_{s,(d,d')}\,\mathrm{d}{r}_{\xi}\,,
%   \label{eq:avgscrN}  % Completely incorrect label!
% \end{equation}
% %%
% such that $l_{d,d'\ne d} = \partial A_d$ are the (hyper)edges of face $A_d$.
% Quadratures of sufficient order can be used to replace the integrals.  Because~\eqref{eq:avgscrN} only involves terms normal to the face only the normal components of $\avg{\NrmT}$ can be determined. However, this is all that is required for hyperbolic conservation laws. If required, transverse components of $\avg{\NrmT}$ can be computed directly or averaged from the normal components of $\avg{\NrmT}$ on nearby faces in orthogonal directions.

% For each edge, the same integrals of $\vec{\Nc_s}$ over the edge appear for the integral over each face adjacent to that edge but with opposite signs.  Therefore, the integration of $\avg{\NrmT}$ over the complete cell volume is zero as long as the integrals of $\vec{\Nc_s}$ are approximated with the same quadrature formulas wherever they appear.

% The form of $\Nc_{s,(d,d')}$ is given explicitly by
% %%
% \begin{equation}
%   \Nc_{s,(d,d')} = \frac{1}{D - 1}
%   \det((\vecnabla_{\xi}\vecx)^T(d|\ebold^s)(d'|\vecx))\,,
%   \label{eq:pntscrN}
% \end{equation}
% %%
% where $\mathrm{A}(p|\vbrm)$ denotes a modification of matrix $\mathrm{A}$ by replacing row $p$ with vector $\vbrm$.

% Note that the preceding discussion on freestream preservation concerns only the flux divergence.
% The treatment of source terms is straightforward on the mapped grid, as long as the control volumes are stationary, since freestream preservation concerns do not arise.

\section{Grid Metrics}
For a grid transformation, a mapping such that $\vec{\xi}(\vec{x})$, along with an inverse mapping $\vec{x}(\vec{\xi})$ must exist. The grid metrics are the set of directional derivatives, which define the grid mapping and are needed for calculation. The metrics and inverse metrics are defined respectively as
\begin{center}
  \begin{align*}
    \text{Inverse Grid Metrics:} \qquad&
    \vecnabla_{\xi} \vecx \equiv \fpartial{(x_1,x_2,x_3)}{(\xi_1,\xi_2,\xi_3)}{1} =
    \begin{bmatrix}
      \displaystyle\fpartial{x_1}{\xi_1}{1} & \displaystyle\fpartial{x_1}{\xi_2}{1} & \displaystyle\fpartial{x_1}{\xi_3}{1}\\[2.5ex]
      \displaystyle\fpartial{x_2}{\xi_1}{1} & \displaystyle\fpartial{x_2}{\xi_2}{1} & \displaystyle\fpartial{x_2}{\xi_3}{1}\\[2.5ex]
      \displaystyle\fpartial{x_3}{\xi_1}{1} & \displaystyle\fpartial{x_3}{\xi_2}{1} & \displaystyle\fpartial{x_3}{\xi_3}{1}
    \end{bmatrix}, \\
    \text{Grid Metrics:} \qquad&
    \vecnabla_{x} \vecxi \equiv \fpartial{(\xi_1,\xi_2,\xi_3)}{(x_1,x_2,x_3)}{1} = 
    \begin{bmatrix}
      \displaystyle\fpartial{\xi_1}{x_1}{1} & \displaystyle\fpartial{\xi_1}{x_2}{1} & \displaystyle\fpartial{\xi_1}{x_3}{1}\\[2.5ex]
      \displaystyle\fpartial{\xi_2}{x_1}{1} & \displaystyle\fpartial{\xi_2}{x_2}{1}& \displaystyle\fpartial{\xi_2}{x_3}{1}\\[2.5ex]
      \displaystyle\fpartial{\xi_3}{x_1}{1} & \displaystyle\fpartial{\xi_3}{x_2}{1}& \displaystyle\fpartial{\xi_3}{x_3}{1}
    \end{bmatrix}\,.
  \end{align*}
\end{center}
By definition it is known that
% \begin{equation}
%   \vecnabla_{\xi} \vecx  \vecnabla_{x} \vecxi  = \Ibrm,
% \end{equation}
% and thus the relation
\begin{equation*}
  \vecnabla_{x} \vecxi = {(\vecnabla_{\xi} \vecx)}^{-1}\,,
\end{equation*}
the grid metric Jacobian is defined as 
\begin{equation*}
  J \equiv \det ( \vecnabla_{\xi} \vecx )\,,
\end{equation*}
and
\begin{equation*}
  \NrmT = J \vecnabla_{x} \vecxi\,.
\end{equation*}
% \begin{eqnarray}
%   J & \equiv & \det ( \vecnabla_{\xi} \vecx )\\
%   & = & \  \fpartial{x_1}{\xi_1}{1} \ \fpartial{x_2}{\xi_2}{1} \  \displaystyle\fpartial{x_3}{\xi_3}{1} \ - \  \fpartial{x_1}{\xi_1}{1} \ \fpartial{x_2}{\xi_3}{1} \  \displaystyle\fpartial{x_3}{\xi_2}{1} \ - \  \fpartial{x_1}{\xi_2}{1} \ \fpartial{x_2}{\xi_1}{1} \ \  \displaystyle\fpartial{x_3}{\xi_3}{1} \nonumber\\
%   &\  & \ + \ \fpartial{x_1}{\xi_2}{1} \ \fpartial{x_2}{\xi_3}{1} \  \displaystyle\fpartial{x_3}{\xi_1}{1} \ + \   \fpartial{x_1}{\xi_3}{1} \ \fpartial{x_2}{\xi_1}{1} \  \displaystyle\fpartial{x_3}{\xi_2}{1} \ -\  \fpartial{x_1}{\xi_3}{1} \ \fpartial{x_2}{\xi_2}{1} \  \displaystyle\fpartial{x_3}{\xi_1}{1} \nonumber \,.
% \end{eqnarray}
% For a smooth 2-D mapping, the metric Jacobian reduces to 
% \begin{equation}
%   J =  \fpartial{x_1}{\xi_1}{1} \ \fpartial{x_2}{\xi_2}{1} -
%   \fpartial{x_1}{\xi_2}{1} \ \fpartial{x_2}{\xi_1}{1} \,.\nonumber
% \end{equation}
% As seen by Guzik et al.\cite{Guzik2015} a matrix metric transformation term, $\NrmT$, can be derived. Assuming the coordinates are time invariant, this is defined as
% \begin{equation}
%   \NrmT = J \vecnabla_{x} \vecxi
% \end{equation}
% % or in a more accessible form
% % \begin{equation}
% %   \NrmT = J {(\vecnabla_{\xi} \vecx)}^{-1}
% % \end{equation}

While these quantities are known at any point, the average $\avg{\NrmT_d}$ on a face, as used in \eqnref{semidiscrDiv.map}, is determined using a more elaborate process to ensure freestream preservation~\cite{Guzik2015}.
Ultimately, $\avg{\NrmT_d}$ is computed on a face normal to direction $d$ from a line integral of $\Nc_{s,(d,d')}, d' \ne d$ around the edges of the face.
%%
$\Nc_{s,(d,d')}$ is defined as
\begin{equation}
  \Nc_{s,(d,d')} = \frac{1}{D - 1}
  \det((\vecnabla_{\xi}\vecx)^T(d|\ebold^s)(d'|\vecx))\,,
  \label{eq:pntscrN}
\end{equation}
%%
where $\mathrm{A}(p|\vbold)$ denotes a modification of matrix $\mathrm{A}$ by replacing row $p$ with vector $\vbold$.
At interfaces of mesh resolution, $\Nc_s$ is not continuous across the face and a one-order loss of accuracy can be observed.
% For this reason, it is desired to have the metrics computed to $O(\Delta x^{p+1})$ accuracy in order for $\avg{\NrmT_d}$ to be accurate to $O(\Delta x^p)$ everywhere.

\section{Useful Transformations}

%% add some matrix expansions that show up a couple times
\newcommand{\gradMat}[6]{\begin{bmatrix} %{u}{v}{w}{x}{y}{z}
    \fpartial{#1}{#4}{1} & \fpartial{#1}{#5}{1} & \fpartial{#1}{#6}{1} \\[1.5ex]
    \fpartial{#2}{#4}{1} & \fpartial{#2}{#5}{1} & \fpartial{#2}{#6}{1} \\[1.5ex]
    \fpartial{#3}{#4}{1} & \fpartial{#3}{#5}{1} & \fpartial{#3}{#6}{1} \\[1.5ex]
  \end{bmatrix}}

\newcommand{\gradMatT}[6]{\begin{bmatrix} %{u}{v}{w}{x}{y}{z}
    \fpartial{#1}{#4}{1} & \fpartial{#2}{#4}{1} & \fpartial{#3}{#4}{1} \\[1.5ex]
    \fpartial{#1}{#5}{1} & \fpartial{#2}{#5}{1} & \fpartial{#3}{#5}{1} \\[1.5ex]
    \fpartial{#1}{#6}{1} & \fpartial{#2}{#6}{1} & \fpartial{#3}{#6}{1} \\[1.5ex]
  \end{bmatrix}}

\newcommand{\gradXiMat}[3]{\gradMat{#1}{#2}{#3}{\xi}{\eta}{\zeta}}

\newcommand{\gradVec}[4]{\begin{bmatrix} %{\phi}{x}{y}{z}
    \fpartial{#1}{#2}{1} \\[1.5ex]
    \fpartial{#1}{#3}{1} \\[1.5ex]
    \fpartial{#1}{#4}{1} \\[1.5ex]
  \end{bmatrix}}

\newcommand{\nablaXi}{\begin{bmatrix}
                      \fpartial{}{\xi}{1} \\[1.5ex]
                      \fpartial{}{\eta}{1} \\[1.5ex]
                      \fpartial{}{\zeta}{1}
                    \end{bmatrix}}

\newcommand{\metric}{\gradMat{\xi}{\eta}{\zeta}{x}{y}{z}}

\newcommand{\metricT}{\gradMatT{\xi}{\eta}{\zeta}{x}{y}{z}}

\newcommand{\invmetric}{\gradMat{x}{y}{z}{\xi}{\eta}{\zeta}}

\newcommand{\nablaXmapped}{ \begin{bmatrix}
    \fpartial{\xi}{x}{1}\fpartial{}{\xi}{1} + \fpartial{\eta}{x}{1}\fpartial{}{\eta}{1} + \fpartial{\zeta}{x}{1}\fpartial{}{\zeta}{1} \\[1.5ex]
    \fpartial{\xi}{y}{1}\fpartial{}{\xi}{1} + \fpartial{\eta}{y}{1}\fpartial{}{\eta}{1} + \fpartial{\zeta}{y}{1}\fpartial{}{\zeta}{1} \\[1.5ex]
    \fpartial{\xi}{z}{1}\fpartial{}{\xi}{1} + \fpartial{\eta}{z}{1}\fpartial{}{\eta}{1} + \fpartial{\zeta}{z}{1}\fpartial{}{\zeta}{1} \\[1.5ex]
  \end{bmatrix}}

\newcommand{\Uvec}{\begin{bmatrix}
                   u \\[1.5ex] v \\[1.5ex] w
                 \end{bmatrix}}

For mapped solutions, as is seen in the conservative form, problems are solved in terms of conservative variables $JU$ in computational coordinates $\vec{\xi}$. This transformation however in no way changes the primitive variables $U$, which are still in terms of physical (likely Cartesian) space $\vec{x}$. The main difference encountered in mapping is that all derivative related operators undergo a change of variables. Expressing these in terms of the nabla operators, the chain rule yields $\vecnabla_x = (\vecnabla_x \xi) \vecnabla_{\xi} = \NJ \vecnabla_{\xi}$ which will be used frequently. Keep in mind this is still an operator, and for the sake of completeness this expanded as
\begin{align*}
  \vecnabla_x &= \NJ \vecnabla_{\xi} \\
  \begin{bmatrix}
    \fpartial{}{x}{1} \\[1.5ex]
    \fpartial{}{y}{1} \\[1.5ex]
    \fpartial{}{z}{1}
  \end{bmatrix}
           & =  \metricT \nablaXi  = \nablaXmapped
\end{align*}
which is seen to simply be the chain rule expansion of each term. The transposed of the mapped nabla operator is equally valid where $\NJ \vecnabla_{\xi} = \vecnabla^{\rmT}_{\xi} \NtJ$, however the transpose I find slightly more awkward to use. Additionally it is easily interpreted wrong as a gradient of $\NrmT$ which leads to confusion. To avoid that problem it is noteworthy that any mixed derivative $\fpartial{}{\xi_i}{1} \big(\fpartial{\xi_j}{x_k}{1} \big)$ must be equal to zero, and thus $\vecnabla_{\xi} \cdot \NtJ = \vec{0}$

\subsubsection{Tensor Notation}
  Some tensor notation review

  The dot product (or inner product) operates on two vectors and results in a scalar $a \cdot b = \sum_i a_i b_i$
  
  The double dot product is an inner prodoct that operates on two matrices and returns a scalar. It is denoted by $:$ and is defined as $A:B = \sum_i \sum_j A_{i,j} B_{i,j}$


\subsection{Divergence}
In physical space the divergence of a vector $u$ is expressed as
\begin{align*}
  \vecnabla_x \cdot \vec{u} 
  &= \fpartial{u_i}{x_i}{1} \\
  &= \fpartial{\xi_j}{x_i}{1} \fpartial{u_i}{\xi_j}{1} \\
  &= \bigg(\NJ \vecnabla_{\xi} \bigg) \cdot \vec{u} \\
  &= \NJ : \vecnabla_{\xi} \vec{u}
\end{align*}
Expanding this yields
\begin{align*}
  \vecnabla_x \cdot \vec{u}
  &= \fpartial{u}{x}{1} + \fpartial{v}{y}{1} + \fpartial{w}{z}{1} \\
  &= \bigg(\NJ \vecnabla_{\xi} \bigg) \cdot \vec{u}
    = \nablaXmapped \cdot \Uvec \\
  &= \NJ : \vecnabla_{\xi} \vec{u}
    = \metricT :  \gradXiMat{u}{v}{w}\\
  &= \bigg( \fpartial{\xi}{x}{1} \fpartial{u}{\xi}{1} + \fpartial{\eta}{x}{1} \fpartial{u}{\eta}{1} + \fpartial{\zeta}{x}{1} \fpartial{u}{\zeta}{1} \bigg) +
    \bigg( \fpartial{\xi}{y}{1} \fpartial{v}{\xi}{1} + \fpartial{\eta}{y}{1} \fpartial{v}{\eta}{1} + \fpartial{\zeta}{y}{1} \fpartial{v}{\zeta}{1} \bigg) +
    \bigg( \fpartial{\xi}{z}{1} \fpartial{w}{\xi}{1} + \fpartial{\eta}{z}{1} \fpartial{w}{\eta}{1} + \fpartial{\zeta}{z}{1} \fpartial{w}{\zeta}{1} \bigg)
\end{align*}

An additional form that is rather useful for flux calculation is
\begin{align*}
  \vecnabla_x \cdot \vec{u} 
  &= \vecnabla_{\xi} \cdot \bigg(\NtJ \vec{u} \bigg) \\
  &= \bigg( \cancelto{0}{ \vecnabla_{\xi} \cdot \NJ } \bigg) \cdot \vec{u} + \NJ : \bigg( \vecnabla_{\xi} \vec{u} \bigg) \\
  &= \NJ : \vecnabla_{\xi} \vec{u}
\end{align*}

\subsection{Gradient}
In physical space the gradient of a vector $u$ is expressed as
\begin{align*}
  \vecnabla_x \vec{u}
  &= \fpartial{u_i}{x_j}{1} \\
  &= \fpartial{\xi_k}{x_j}{1} \fpartial{u_i}{\xi_k}{1} \\
  &= \bigg( \Big(\NJ \vecnabla_{\xi} \Big) \otimes \vec{u}\bigg)^{\rmT} \\
  &= \big(\vecnabla_{\xi} \vec{u} \big) \Big( \NtJ \Big)
\end{align*}
which is expanded as
\begin{align*}
  \vecnabla_x \vec{u}
  &= \gradMat{u}{v}{w}{x}{y}{z} \\
  %&= \metric \nablaXi \Uvec \\
  %&= \bigg( \Big(\NJ \vecnabla_{\xi} \Big) \otimes \vec{u}\bigg)^{\rmT} = \nablaXmapped \otimes \Uvec ^{\rmT} \\
  &= \big(\vecnabla_{\xi} \vec{u} \big) \Big( \NtJ \Big) = \gradXiMat{u}{v}{w}  \metric \\
  &= 
    \begin{bmatrix}
      \fpartial{\xi}{x}{1}\fpartial{u}{\xi}{1} + \fpartial{\eta}{x}{1}\fpartial{u}{\eta}{1} + \fpartial{\zeta}{x}{1}\fpartial{u}{\zeta}{1} &
      \fpartial{\xi}{y}{1}\fpartial{u}{\xi}{1} + \fpartial{\eta}{y}{1}\fpartial{u}{\eta}{1} + \fpartial{\zeta}{y}{1}\fpartial{u}{\zeta}{1} &
      \fpartial{\xi}{z}{1}\fpartial{u}{\xi}{1} + \fpartial{\eta}{z}{1}\fpartial{u}{\eta}{1} + \fpartial{\zeta}{z}{1}\fpartial{u}{\zeta}{1} \\[1.5ex]
      \fpartial{\xi}{x}{1}\fpartial{v}{\xi}{1} + \fpartial{\eta}{x}{1}\fpartial{v}{\eta}{1} + \fpartial{\zeta}{x}{1}\fpartial{v}{\zeta}{1} &
      \fpartial{\xi}{y}{1}\fpartial{v}{\xi}{1} + \fpartial{\eta}{y}{1}\fpartial{v}{\eta}{1} + \fpartial{\zeta}{y}{1}\fpartial{v}{\zeta}{1} &
      \fpartial{\xi}{z}{1}\fpartial{v}{\xi}{1} + \fpartial{\eta}{z}{1}\fpartial{v}{\eta}{1} + \fpartial{\zeta}{z}{1}\fpartial{v}{\zeta}{1} \\[1.5ex]
      \fpartial{\xi}{x}{1}\fpartial{w}{\xi}{1} + \fpartial{\eta}{x}{1}\fpartial{w}{\eta}{1} + \fpartial{\zeta}{x}{1}\fpartial{w}{\zeta}{1} &
      \fpartial{\xi}{y}{1}\fpartial{w}{\xi}{1} + \fpartial{\eta}{y}{1}\fpartial{w}{\eta}{1} + \fpartial{\zeta}{y}{1}\fpartial{w}{\zeta}{1} &
      \fpartial{\xi}{z}{1}\fpartial{w}{\xi}{1} + \fpartial{\eta}{z}{1}\fpartial{w}{\eta}{1} + \fpartial{\zeta}{z}{1}\fpartial{w}{\zeta}{1} \\[1.5ex]
    \end{bmatrix}
\end{align*}
which is again a simple, although tedious, application of the chain rule

A note on Chombo storage. It commonly make more sense to store $(\vecnabla_{\xi} \vec{u})^T$, and $(\vecnabla_{x} \vec{u})^T$ because each component of $u$ can be stored contiguously which is far more natural as it works for any number of components. Thus the above transformation is more likely used as $(\vecnabla_x \vec{u})^T = \NJ \big(\vecnabla_{\xi} \vec{u} \big)^T$


\subsection{Gradient of a Scalar}
In physical space the gradient of a scalar $\phi$ is expressed as
\begin{align*}
  \vecnabla_x \phi
  &= \fpartial{\phi}{x_j}{1} \\
  &= \fpartial{\xi}{x_j}{1} \fpartial{\phi}{\xi_k}{1} \\
  &= \big(\vecnabla_{\xi} \phi \big)^{\rm{T}} \Big( \NtJ \Big) = \NJ \big(\vecnabla_{\xi} \phi \big) \\
\end{align*}
which is expanded as
\begin{align*}
  \vecnabla_x \phi
  &= \gradVec{\phi}{x}{y}{z} \\
  &= \NJ \big(\vecnabla_{\xi} \phi \big) = \metricT \gradVec{\phi}{\xi}{\eta}{\zeta} \\
  &= 
    \begin{bmatrix}
      \fpartial{\xi}{x}{1}\fpartial{\phi}{\xi}{1} + \fpartial{\eta}{x}{1}\fpartial{\phi}{\eta}{1} + \fpartial{\zeta}{x}{1}\fpartial{\phi}{\zeta}{1} \\[1.5ex]
      \fpartial{\xi}{y}{1}\fpartial{\phi}{\xi}{1} + \fpartial{\eta}{y}{1}\fpartial{\phi}{\eta}{1} + \fpartial{\zeta}{y}{1}\fpartial{\phi}{\zeta}{1} \\[1.5ex]
      \fpartial{\xi}{z}{1}\fpartial{\phi}{\xi}{1} + \fpartial{\eta}{z}{1}\fpartial{\phi}{\eta}{1} + \fpartial{\zeta}{z}{1}\fpartial{\phi}{\zeta}{1} \\[1.5ex]
    \end{bmatrix}
\end{align*}
which is again an application of the chain rule

\subsection{Curl}
In physical space the curl of a vector $u$ is expressed as
\begin{align*}
  \vecnabla_x \times \vec{u}
  &= \epsilon_{ijk}\fpartial{u_k}{x_j}{1} \\
  &= \epsilon_{ijk} \fpartial{\xi_l}{x_j}{1} \fpartial{u_k}{\xi_l}{1} \\
  &= \Big(\NJ \vecnabla_{\xi} \Big) \times \vec{u}
\end{align*}
This unfortunately does not appear to be easily decomposed. In expanded form this becomes
\begin{align*}
  \vecnabla_x \times \vec{u}
  &=
    \begin{bmatrix}
      \fpartial{w}{y}{1} - \fpartial{v}{z}{1} \\[1.5ex]
      \fpartial{u}{z}{1} - \fpartial{w}{x}{1} \\[1.5ex]
      \fpartial{v}{x}{1} - \fpartial{u}{y}{1} 
    \end{bmatrix} \\
  &=
    \begin{bmatrix}
      \big( \fpartial{\xi}{y}{1} \fpartial{w}{\xi}{1} + \fpartial{\eta}{y}{1} \fpartial{w}{\eta}{1} + \fpartial{\zeta}{y}{1} \fpartial{w}{\zeta}{1} \big) -
      \big( \fpartial{\xi}{z}{1} \fpartial{v}{\xi}{1} + \fpartial{\eta}{z}{1} \fpartial{v}{\eta}{1} + \fpartial{\zeta}{z}{1} \fpartial{v}{\zeta}{1} \big) \\[1.5ex]
      \big( \fpartial{\xi}{z}{1} \fpartial{u}{\xi}{1} + \fpartial{\eta}{z}{1} \fpartial{u}{\eta}{1} + \fpartial{\zeta}{z}{1} \fpartial{u}{\zeta}{1} \big) -
      \big( \fpartial{\xi}{x}{1} \fpartial{w}{\xi}{1} + \fpartial{\eta}{x}{1} \fpartial{w}{\eta}{1} + \fpartial{\zeta}{x}{1} \fpartial{w}{\zeta}{1} \big) \\[1.5ex]
      \big( \fpartial{\xi}{x}{1} \fpartial{v}{\xi}{1} + \fpartial{\eta}{x}{1} \fpartial{v}{\eta}{1} + \fpartial{\zeta}{x}{1} \fpartial{v}{\zeta}{1} \big) -
      \big( \fpartial{\xi}{y}{1} \fpartial{u}{\xi}{1} + \fpartial{\eta}{y}{1} \fpartial{u}{\eta}{1} + \fpartial{\zeta}{y}{1} \fpartial{u}{\zeta}{1} \big)
    \end{bmatrix}
\end{align*}


\ifx\havefulldoc\undefined
\end{document}
\fi
