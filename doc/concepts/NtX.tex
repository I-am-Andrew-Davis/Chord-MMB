% Allows compilation of just this section in this directory
\ifx\havefulldoc\undefined
  \newif\iffulldoc
\ifx\havefulldoc\undefined
  \fulldocfalse
\else
  \fulldoctrue
\fi
\documentclass[11pt]{article}

%\usepackage{subfigmat}  % matrices of similar subfigures, aka small mulitples
%                        % use this instead of subfig which can cause conflicts  
\usepackage{graphicx}
\usepackage[rgb]{xcolor}
\usepackage{calc}        % Can add lengths
\usepackage{ifthen}
\usepackage{tikz}
\usepackage{pgfplots}
\usepackage{xspace}      % Spacing after some new commands
\usepackage{amsmath}
\usepackage{amssymb}
\usepackage{bm}
\usepackage{cancel}      % show math cancellation
\usepackage{subfig}
\usepackage{wrapfig}     % Wrap text around figures
%%\usepackage{sidecap}   % Caption on side
\usepackage{caption}     % Caption format
\usepackage[numbers,sort&compress]{natbib}
\usepackage{url}         % Can use \path for textt
%%\usepackage{ctable}    % Tables with footnotes
\usepackage{array}       % Table columns
\usepackage{longtable}   % Table with page breaks
\usepackage{newtxtext,newtxmath}
%\usepackage{mathptmx}    % Times new roman font
%\usepackage{mathpazo}  % Palatino font
\usepackage[hypertexnames=false, pdfborder={0 0 0}]{hyperref}  % Links in the document
\usepackage[pass,paperwidth=8.5in,paperheight=11in]{geometry}  % Papersize
\usepackage{enumitem}
\usepackage{pdfpages}
\usepackage{listings}    % For code listing
\usepackage{verbatim}    % For verbatim input

\pgfplotsset{compat=1.5}

%--Page formatting

\setlength{\topmargin}{0.in}
\setlength{\headheight}{0.in}
\setlength{\headsep}{0.in}
\setlength{\oddsidemargin}{0.in}
\setlength{\evensidemargin}{0.in}
\setlength{\textwidth}{6.5in}
\setlength{\textheight}{9in}
\setlength{\footskip}{10mm}

%--Caption formatting

\captionsetup{font=small,margin=10pt,maxmargin=0.1\linewidth,labelfont=bf,labelsep=quad}

%--Hyperref setup

% All black
\hypersetup {
colorlinks= true,
urlcolor=black,
citecolor=black,
linkcolor=black,
}

%--Source listings formatting

\lstset{language=C++,
  basicstyle=\ttfamily,
%  keywordstyle=\bfseries  % Not necessary for only bold
  columns=fixed,
  basewidth=0.5em,         % ~Same column width as verbatim
%  numbers=left,
  numberstyle=\scriptsize,
%  showstringspaces=false,
  morekeywords={shared,private}
}
%\lstset{
%        language=C++,
%        basicstyle=\scriptsize,
%%       numbers=left,
%%       numberstyle=,
%        stepnumber=1,
%        numbersep=5pt,
%        backgroundcolor=\color{white},
%        showspaces=false,
%        showstringspaces=false,
%        showtabs=false,
%        frame=single,
%        tabsize=2,
%        captionpos=b,
%        breaklines=true,
%        breakatwhitespace=false,
%        escapeinside={\%*}{*)}
%}

%%%%%%%%%%%%%%%%%%%%%%%%%%%%%%%%%%
%%                              %%
%% New commands                 %%
%%                              %%
%%%%%%%%%%%%%%%%%%%%%%%%%%%%%%%%%%

\makeatletter

%--Preliminary environment

\newenvironment{preliminary}%
  {\pagestyle{plain}\pagenumbering{roman}}%
  {\cleardoublepage\pagenumbering{arabic}}

%--Roman numerals

\newcommand{\rmnum}[1]{\romannumeral #1}
\newcommand{\Rmnum}[1]{\expandafter\@slowromancap\romannumeral #1@}

%--Create a \deg command

\DeclareMathSymbol{\circx}{\mathord}{symbols}{"0E}
\let\olddeg=\deg
\renewcommand\deg{\ensuremath{\,^\circx\,}}

%--A nice C++ symbol

\newcommand*{\Cpp}{C\parbox{1em}{\vspace*{-0.6ex}$\scriptstyle\mspace{-1mu}+\mspace{-1mu}+$}\xspace}%

%--This section creates the \un command for units and \E for exponents
%--E.g., 2.9\un{m^{2}} or 2.9\E{-3}\un{m^{2}}
%--Can be used in or out of math environment (math ensured in braces)
%--also a \un* for no spacing

\DeclareMathSymbol{\timesx}{\mathord}{symbols}{"02}
\newcommand{\etimes}{\mskip1.1mu\timesx\mskip0.9mu\relax}
\newcommand{\E}[1]{\ensuremath{\etimes10^{#1}}}

\newmuskip\unitmuskip
\newmuskip\unitpreskip
\unitmuskip=1mu plus 0.3mu              % Change this to suit your preference
\unitpreskip=4mu plus 1mu minus 1mu     % Change this to suit your preference
\def\unitmskip{\penalty10000 \mskip\unitmuskip}%
\DeclareRobustCommand\un{\@ifstar{\@unstar}{\@un}}
\newcommand\@unstar[1]{\leavevmode
  \ensuremath{%
    \begingroup \fam\z@%
    \let\,\unitmskip \unit@PreserveSpaces\@empty #1 \unit@PreserveSpaces%
    \endgroup}}%
\newcommand\@un[1]{\leavevmode%
  \unskip\penalty1000%
  \ensuremath{%
    \@checkfordeg #1\@empty%
    \begingroup \fam\z@%
    \let\,\unitmskip \unit@PreserveSpaces\@empty #1 \unit@PreserveSpaces%
    \endgroup}}
\def\unit@PreserveSpaces#1 {#1\@ifnextchar\unit@PreserveSpaces{\@gobble}%
  {\unitmskip \unit@PreserveSpaces\@empty}}%
\def\@checkfordeg{\@ifnextchar\deg{\@junkrest}{\mskip\unitpreskip\@junkrest}}%
\def\@junkrest#1\@empty{}%

%--Corrections and notes

\newcommand{\mnote}[1]{\marginpar{\textcolor{red}{#1}}}%
\newcommand{\inote}[1]{\textcolor{red}{[#1]}}%
\newcommand{\fnote}[1]{\footnote{\textcolor{red}{#1}}}%

%--Compact enumeration

\newcounter{cmpctenum}
\newenvironment{compactenum}{\begin{list}{\arabic{cmpctenum}.}
  {\usecounter{cmpctenum}
   \setlength{\parsep}{0pt}
   \setlength{\itemsep}{0.5ex}}
  }
{\end{list}}

%--Compact itemize

\newenvironment{compactitem}{\begin{list}{{\footnotesize$\bullet$}}
  {%\setlength{\topsep}{0pt}
   \setlength{\parsep}{0pt}
   \setlength{\itemsep}{0.5ex}
   \setlength{\leftmargin}{1.5em}
   \setlength{\itemindent}{1.0em}}
  }
{\end{list}}

%--Compact itemize for function arguments

\newenvironment{argitem}{\begin{list}{{\footnotesize$\circ$}}
  {\setlength{\topsep}{0pt}
    \setlength{\parsep}{0pt}
   \setlength{\itemsep}{0.5ex}
   \setlength{\leftmargin}{2em}
   \setlength{\itemindent}{-1em}}
  }
{\end{list}}

%--Figure location which changes if full or chapter compilation

\iffulldoc
  \newcommand*{\figpre}[1]{figures/#1}
  \newcommand*{\drawpre}[1]{drawings/#1}
\else
  \newcommand*{\figpre}[1]{../figures/#1}
  \newcommand*{\drawpre}[1]{../drawings/#1}
\fi

%--Consistent references

\newcommand{\figref}[1]{Figure~\ref{#1}} 
\newcommand{\tabref}[1]{Table~\ref{#1}} 
\newcommand{\secref}[1]{Section~\ref{#1}} 
\newcommand{\chapref}[1]{Chapter~\ref{#1}} 
\newcommand{\appref}[1]{Appendix~\ref{#1}} 

%--Commonly used math/Chombo notation

\newcommand{\ibold}{{\boldsymbol{i}}}
\newcommand{\jbold}{{\boldsymbol{j}}}
\newcommand{\kbold}{{\boldsymbol{k}}}
\newcommand{\ebold}{{\boldsymbol{e}}}
\newcommand{\pbold}{{\boldsymbol{p}}}
\newcommand{\ubold}{{\boldsymbol{u}}}
\newcommand{\vbold}{{\boldsymbol{v}}}
\newcommand{\xbold}{{\boldsymbol{x}}}
\newcommand{\bbold}{{\boldsymbol{b}}}
\newcommand{\hbold}{{\boldsymbol{h}}}
\newcommand{\Ubold}{{\boldsymbol{U}}}
\newcommand{\Wbold}{{\boldsymbol{W}}}
\newcommand{\Nbold}{{\boldsymbol{N}}}
\newcommand{\Xbold}{{\boldsymbol{X}}}
\newcommand{\xibold}{{\boldsymbol{\xi}}}
\newcommand{\nablabold}{{\boldsymbol{\nabla}}}
\newcommand{\Deltabold}{{\boldsymbol{\Delta}}}
\newcommand{\phibold}{{\boldsymbol{\phi}}}
\newcommand{\eboldd}{{\ebold^d}}
\newcommand{\zerobold}{{\boldsymbol{0}}}
\newcommand{\Dim}{{\mathbf{D}}}
\newcommand{\NboldT}{{\Nbold^T}}
\newcommand{\NrmT}{{\text{N}}^T}
\newcommand{\Nrm}{{\text{N}}}
\newcommand{\Frm}{{\text{F}}}
\newcommand{\Wrm}{{\text{W}}}
\newcommand{\Urm}{{\text{U}}}
\newcommand{\Fbrm}{{\text{\bfseries F}}}
\newcommand{\Gbrm}{{\text{\bfseries G}}}
\newcommand{\Sbrm}{{\text{\bfseries S}}}
\newcommand{\Ubrm}{{\text{\bfseries U}}}
\newcommand{\Wbrm}{{\text{\bfseries W}}}
\newcommand{\Nc}{{\mathcal{N}}}
\newcommand{\vecnabla}{{\vec{\nabla}}}
\newcommand{\vecx}{{\vec{x}}}
\newcommand{\vecxi}{{\vec{\xi}}}

% Full partial {numerator}{denominator}{degree}
\newcommand*{\fpartial}[3]{\ifnum#3=1\ensuremath{\frac{\partial #1}{\partial #2}}\else\ensuremath{\frac{\partial^{#3} #1}{\partial #2^{#3}}}\fi}
% Full partial {numerator}{denominator}{degree} with extra spacing for denom
\newcommand*{\fpartialspc}[3]{\ifnum#3=1\ensuremath{\frac{\partial #1}{\partial #2}}\else\ensuremath{\frac{\partial^{#3} #1}{\partial #2^{\,#3}}}\fi}
% Full mixed partial {numerator}{denom1}{degree1}{denom2}{degree2}
\newcounter{npdeg}
\newcommand*{\fmpartial}[5]{\setcounter{npdeg}{#3}\addtocounter{npdeg}{#5}\ensuremath{\frac{\partial^{\thenpdeg} #1}{\partial{#2}\ifnum#3>1^{#3}\fi\partial{#4}\ifnum#5>1^{#5}\fi}}}
% Cell step subscript (argument is '+n'+ or '-n')
\newcommand*{\hstep}[1]{\ensuremath{{\ibold#1\eboldd}}}
% Cell half step subscript (argument is '+' or '-')
\newcommand*{\hhstep}[1]{\ensuremath{{\ibold#1\frac{1}{2}\eboldd}}}
% Cell half step subscript (argument is '+' or '-'), second is number of halves
\newcommand*{\hhstepn}[2]{\ensuremath{{\ibold#1\frac{#2}{2}\eboldd}}}
% Average angle brackets
\newcommand*{\avg}[1]{\ensuremath{\langle #1\rangle}}

% \vt = v_\theta
\newcommand*{\vt}{\ensuremath{v_\theta}}

%--Spacing

\newcommand{\zapspace}{\topsep=1pt\partopsep=1pt\itemsep=1pt\parskip=2pt}

\makeatother

%--Float fractions

\renewcommand{\topfraction}{0.9}
\renewcommand{\bottomfraction}{0.9}
\renewcommand{\textfraction}{0.1}
\renewcommand{\floatpagefraction}{0.9}

  \begin{document}
\fi

\begin{center}
  \large\bfseries Considerations for the volume flux, $\avg{\NrmT_d\vec{\chi}}$, at block periodic boundaries that overlap with refinement interfaces
\end{center}

The volume flux, $\avg{\NrmT_d \vec{\chi}}$, is used to define $\avg{J}$:
%%
\begin{align*}
  \avg{J} &= \frac{1}{h^D} \int \limits_{x(V_\ibold)} \mathrm{d} \vecx\\
          &= \frac{1}{h^D} \int \limits_{x(V_\ibold)} \vecnabla_x \cdot \frac{\vecx}{D} \mathrm{d} \vecx\\
          &= \frac{1}{h^D} \int \limits_{V_\ibold} \vecnabla_{{\xi}} \cdot (\NrmT\vec{\chi}) \mathrm{d} \vecxi\,,
\end{align*}
%%
where $\vec{\chi}(\,\vecxi\,)\!=\! \frac{\vecx(\vecxi)}{D}$.  The volume flux is used to construct $\avg{J}$ and to define the snapback flux during regrid operations.  To ensure freestream preservation, it must be consistent across the grid.  Problems occur at periodic boundaries because the value of $\vecx$ changes across the periodic jump.  $\avg{\NrmT_d\vec{\chi}}$ does not have a consistent definition.  During application of the face register, where the sum of the evaluation of $\avg{\NrmT_d\vec{\chi}}^{\ell+1}$ on the fine faces is used to overwrite $\avg{\NrmT_d{\vec{\chi}}}^{\ell}$ on the coarse face across a periodic boundary, the change in $\vecx$ must be considered.  In the following, this change is denoted as $\vec{\delta}$.

On a face, the average of $\vecx$ is given by:
%%
\begin{equation}
  \avg{\vecx}_{\hhstep{+}} = \vecx_{\hhstep{+}} + \frac{h^2}{24} \Delta^{\perp,d} {\vecx}_{\hhstep{+}}\,,
  \label{eq:avgx}
\end{equation}
%%
where the $\Delta^{\perp,d}$ is a second-order accurate Laplacian computed only in directions orthogonal to $d$, i.e.,
%%
\begin{equation}
  \Delta^{\perp,d} = \Delta - \fpartial{}{\xi_d}{2}\,. \nonumber
\end{equation}
%%
At the other side of a periodic boundary,
\begin{equation}
  \avg{\vecx + \vec{\delta}}_{\hhstep{+}} = \vecx_{\hhstep{+}} + \vec{\delta}_{\hhstep{+}} + \frac{h^2}{24} \Delta^{\perp,d} (\vecx + \vec{\delta})_{\hhstep{+}}\,.
  \label{eq:avgxd}
\end{equation}
%%
Since the faces are conforming, we know that $\Delta^{\perp,d} {\vecx}_{\hhstep{+}} = \Delta^{\perp,d} (\vecx + \vec{\delta})_{\hhstep{+}}$.  Therefore, subtracting \eqref{eq:avgx} from \eqref{eq:avgxd} yields
%%
\begin{equation}
\avg{\vecx + \vec{\delta}}_{\hhstep{+}} = \avg{\vecx}_{\hhstep{+}} + \vec{\delta}_{\hhstep{+}}\,.
\end{equation}
%%

In a similar fashion, the product rule is used to find the average volume flux,
%%
\begin{equation}
  \avg{\NrmT_d\vec{\chi}}_{\hhstep{+}} = \avg{\NrmT_d}_{\hhstep{+}}\avg{\vec{\chi}}_{\hhstep{+}} + \frac{h^2}{12}\sum_{d'\ne d}\fpartial{\NrmT_d}{\xi_{d'}}{1}\fpartial{\vec{\chi}}{\xi_{d'}}{1} + O(h^4)\,,
  \label{eq:avgNtX}
\end{equation}
%%
and
%%
\begin{equation}
  \avg{\NrmT_d(\vec{\chi} + \vec{\delta}/D)}_{\hhstep{+}} = \avg{\NrmT_d}_{\hhstep{+}}\avg{\vec{\chi} + \vec{\delta}/D}_{\hhstep{+}} + \frac{h^2}{12}\sum_{d'\ne d}\fpartial{\NrmT_d}{\xi_{d'}}{1}\fpartial{\vec{\chi} + \vec{\delta}/D}{\xi_{d'}}{1} + O(h^4)\,,
  \label{eq:avgNtXd}
\end{equation}
%%
Because the faces are conforming,
%%
\begin{equation*}
  \fpartial{\vec{\chi}}{\xi_{d'}}{1} = \fpartial{\vec{\chi} + \vec{\delta}/D}{\xi_{d'}}{1}
\end{equation*}
%%
and subtracting \eqref{eq:avgNtX} from \eqref{eq:avgNtXd} yields
%%
\begin{align*}
  \avg{\NrmT_d(\vec{\chi} + \vec{\delta}/D)}_{\hhstep{+}} &= \avg{\NrmT_d\vec{\chi}}_{\hhstep{+}} + \avg{\NrmT_d}_{\hhstep{+}}\avg{\vec{\chi} + \vec{\delta}/D}_{\hhstep{+}} - \avg{\NrmT_d}_{\hhstep{+}}\avg{\vec{\chi}}_{\hhstep{+}}\\
  &= \avg{\NrmT_d\vec{\chi}}_{\hhstep{+}} +  \frac{\avg{\NrmT_d}_{\hhstep{+}}}{D}\left[\avg{\vecx + \vec{\delta}}_{\hhstep{+}} - \avg{\vecx}_{\hhstep{+}}\right]\\
  &= \avg{\NrmT_d\vec{\chi}}_{\hhstep{+}} +  \frac{\avg{\NrmT_d}_{\hhstep{+}}}{D}\left[\avg{\vecx}_{\hhstep{+}} + \vec{\delta}_{\hhstep{+}} - \avg{\vecx}_{\hhstep{+}}\right]\\
  &= \avg{\NrmT_d\vec{\chi}}_{\hhstep{+}} +  \avg{\NrmT_d}_{\hhstep{+}}\frac{\vec{\delta}_{\hhstep{+}}}{D}\,.
\end{align*}
%%
In the above, use $\avg{\NrmT_d}_{\hhstep{+}}^{\ell}$ that is defined as a sum from the finer mesh.

\ifx\havefulldoc\undefined
\end{document}
\fi